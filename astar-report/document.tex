\documentclass{article}

\usepackage[czech, english]{babel} \usepackage[T1]{fontenc} % pouzije EC fonty
\usepackage[utf8]{inputenc}
\usepackage{gensymb}
\usepackage{graphicx}
\usepackage{xfrac}

\begin{document}



\title{A* Algorithm - Report}
\author{Nicolas Boileau, Simon Stastny}

\maketitle

\section{General A* algorithm}

We decided to create own own implementation of A* algorithm for better learning
outome. Our general A* algorithm implementation is found in the package
\texttt{edu.ntnu.simonst.tdt4136.astar} and contains following classes:

\begin{itemize}
  \item Class \texttt{BestFirstSearch} with general A* algorithm implementation
  \item Class \texttt{SearchNode} for general search-node used in the algotithm
  \item Class \texttt{SearchState} for general search-state used in the
  algotithm
  \item Class \texttt{Fringe} used to store nodes in agenda (list of unexpanded
  nodes)
\end{itemize} 

Apart from those general classes, this package contains class \texttt{App} which
is used to run both puzzles from command-line using maven.

To run the puzzles, please use maven command \texttt{mvn exec:java} inside the
project folder, or run the project within NetBeans (maven enabled).

\section{Fractions puzzle}

\subsection{Initial state of the puzzle}

The initial state of the puzzle is state identified by permutation
\texttt{123456789} which represents following fraction:
\[
 \frac{1234}{56789}
\]

\subsection{Description of a goal state}

This puzzle involves finding fractions equal to fractions: 

\[
 \frac{1}{2}\quad\frac{1}{3}\quad\frac{1}{4}\quad\frac{1}{5}\quad\frac{1}{6}\quad\frac{1}{7}\quad
 \frac{1}{8}\quad\frac{1}{9}
\]

For example for the first fraction (\sfrac{1}{2}) the goal state would have
permutation \texttt{796215384}, because following two fractions are equal:
\[\frac{7692}{15384} = \frac{2^2 \times 3 \times 641}{2^2 \times 3 \times 641} =
\frac{1}{2}\].

Similarily for other fractions there exist other permutations of
\texttt{123456789} which represent fractions equal to them respectively.

\subsection{Method of assessing arc costs}
Since we are not really interested in the path to the goal node as much as we
are interested in the goal-node's state itself, we are not concerned about cost
of the solution (lenght of the path).

For this reason, the arc cost for transition from one state to another is fixed
value 1.

\subsection{Heuristic function description}

As long as states are representing fractions, i.e. numbers, the heuristic

A clear, concise description (using mathematical expressions and text) of the
heuristic function (h) used to solve the puzzle.
\subsection{Successor generation procedure}
A thorough description of the procedure used to generate successor states when
expanding a node.
\subsection{Overview description of a solution}
An overview description of a solution (i.e. path from start to goal) found by
A*. The sequence of search states from the start to the goal node must be
presented along with a brief summary (in Norwegian or English) of the main
state-to-state transitions within sequence.

\section{Checkers puzzle}

1. The initial state of the puzzle.
2. A general description of a goal state.
3. The method of assessing arc costs. It may be a xed value or a more complex procedure.
4. A clear, concise description (using mathematical expressions and text) of the heuristic function (h)
used to solve the puzzle.
5. A thorough description of the procedure used to generate successor states when expanding a node.
6. An overview description of a solution (i.e. path from start to goal) found by A*. The sequence of search
states from the start to the goal node must be presented along with a brief summary (in Norwegian or
English) of the main state-to-state transitions within sequence.


\end{document}
