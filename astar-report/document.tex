\documentclass{article}

\usepackage[czech, english]{babel} \usepackage[T1]{fontenc} % pouzije EC fonty
\usepackage[utf8]{inputenc}
\usepackage{gensymb}
\usepackage{graphicx}
\usepackage{xfrac}

\begin{document}



\title{A* Algorithm - Report}
\author{Nicolas Boileau, Simon Stastny}

\maketitle

\section{General A* algorithm}

We decided to create own own implementation of A* algorithm for better learning
outome. Our general A* algorithm implementation is found in the package
\texttt{edu.ntnu.simonst.tdt4136.astar} and contains following classes:

\begin{itemize}
  \item Class \texttt{BestFirstSearch} with general A* algorithm implementation
  \item Class \texttt{SearchNode} for general search-node used in the algotithm
  \item Class \texttt{SearchState} for general search-state used in the
  algotithm
  \item Class \texttt{Fringe} used to store nodes in agenda (list of unexpanded
  nodes)
\end{itemize} 

Apart from those general classes, this package contains class \texttt{App} which
is used to run both puzzles from command-line using maven.

To run the puzzles, please use maven command \texttt{mvn exec:java} inside the
project folder, or run the project within NetBeans (maven enabled).

\section{Fractions puzzle}

\subsection{Initial state of the puzzle}

The initial state of the puzzle is state identified by permutation
\texttt{123456789} which represents following fraction:
\[
 \frac{1234}{56789}
\]

\subsection{Description of a goal state}

This puzzle involves finding fractions equal to fractions: 

\[
 \frac{1}{2}\quad\frac{1}{3}\quad\frac{1}{4}\quad\frac{1}{5}\quad\frac{1}{6}\quad\frac{1}{7}\quad
 \frac{1}{8}\quad\frac{1}{9}
\]

For example for the first fraction (\sfrac{1}{2}) the goal state would have
permutation \texttt{796215384}, because following two fractions are equal:
\[\frac{7692}{15384} = \frac{2^2 \times 3 \times 641}{2^2 \times 3 \times 641} =
\frac{1}{2}\].

Similarily for other fractions there exist other permutations of
\texttt{123456789} which represent fractions equal to them respectively.

\subsection{Method of assessing arc costs}
Since we are not really interested in the path to the goal node as much as we
are interested in the goal-node's state itself, we are not concerned about cost
of the solution (lenght of the path). However, we are still trying to work in an
optimal way and thus we would like to get to the goal taking as few steps as
possible.

For this reason, the arc cost for transition from one state to another is fixed
value 1 (in the inner code 1 000 000, because of precision problems related to
heuristic function), hence the total cost of path from root to goal is equal to
number of steps we have taken.

\subsection{Heuristic function description}

As long as states are representing fractions, i.e. numbers, the heuristic
for estimating cost from current state to goal state could be something so
simple as mathematical difference of those two fractions.

Since all the fractions possible in this puzzle have values greater or equal to
0.0124943046625829 and lesser of equal to 0.8, the mathematical difference of
any two of them would be a double value somewhere between 0 and 1.

\[
 \forall x \in possible fractions, \frac{1234}{98765} \leq x \leq
 \frac{9876}{12345}
\]

We need to map those differences to cardinal numbeers in a way which preserves
the order and where no difference would be equal to 0, the smallest difference
would be equal to 1 and big differences would be map to big numbers.

Our solution counts the double value of difference of those two fractions,
multiplies the value by 1 000 000 and subtracts 10. This results into the
smallest difference being assigned estimate of 1, and bigger differences scaled
up to somewhere below nearly 800 000.

Those big numbers are the reason why the arc costs are multiplied by 1 000 000.

\subsection{Successor generation procedure}

Successor nodes are generated from current node with a simple switch operator.
This switch operator takes the permutation of the node's state (which is a
permutation of digits from 1 to 9) and switches two digits to make a new state.
Since the first digit can be switched with 8 others to make a new permutations,
the second digit with 7 others (because switching with the first permutation
was done already), third one with 6 others\ldots, this generates 36
(8+7+\ldots+1) new permutations, and each one of them is assigned to a successor
node.

This leaves us with 36 successor nodes to each node. At least one of those
successor nodes is already closed (because it is current node's parent). We
evaluate the remaining nodes' cost estimates and decide which node to expand in
the next turn.

\subsection{Overview description of a solution}

This puzzle is searching for solutions fo 8 different fractions. Those solutions
are listed in the subsections below.

\subsubsection{Solution for \sfrac{1}{2}}

For the first fraction (\sfrac{1}{2}) the solution permutation 729314586 was
found. This is a solution, because the fraction it represents is equal to
goal-state's fraction.

\[\frac{7692}{15384} = \frac{2^2 \times 3 \times 641}{2^2 \times 3 \times 641} =
\frac{1}{2}\].

This solution was found in just 5 steps listed here.

\begin{itemize}
  \item \textbf{root} state: 1234/56789 (0.02172956) 
  \item state: 5234/16789 (0.311751742)
  \item state: 7234/16589 (0.436072096)
  \item state: 7294/16583 (0.439848037)
  \item state: 7294/13586 (0.536876196)
  \item \textbf{goal} state: 7293/14586 (0.5)
\end{itemize}

\subsubsection{Solution for \sfrac{1}{3}}

For the first fraction (\sfrac{1}{3}) the solution permutation 582317469 was
found. This is a solution, because the fraction it represents is equal to
goal-state's fraction.

\[\frac{5823}{17469} = \frac{3^2 \times 647}{3^3 \times 647} =
\frac{1}{3}\].

This solution was found in just 6 steps listed here.

\begin{itemize}
  \item \textbf{root} state: 1234/56789 (0.02172956) 
  \item state: 5234/16789 (0.311751742)
  \item state: 5234/17689 (0.295890101)
  \item state: 5834/17629 (0.330931987)
  \item state: 5834/17269 (0.337830795)
  \item state: 5824/17369 (0.335310035)
  \item \textbf{goal} state: 5823/17469 (0.333333333)
\end{itemize}

\subsubsection{Solution for \sfrac{1}{4}}

For the first fraction (\sfrac{1}{4}) the solution permutation 579623184 was
found. This is a solution, because the fraction it represents is equal to
goal-state's fraction.

\[\frac{5796}{23184} = \frac{2^2 \times 3^2 \times 7 \times 23}{2^4 \times 3^2 \times 7 \times 23}
= \frac{1}{4}\].

This solution was found in just 6 steps listed here.

\begin{itemize}
  \item \textbf{root} state: 1234/56789 (0.02172956) 
  \item state: 1534/26789 (0.057262309)
  \item state: 7534/26189 (0.287678033)
  \item state: 5734/26189 (0.218946886)
  \item state: 5736/24189 (0.237132581)
  \item state: 5796/24183 (0.239672497)
  \item \textbf{goal} state: 5796/23184 (0.25)
\end{itemize}

\subsubsection{Solution for \sfrac{1}{5}}

For the first fraction (\sfrac{1}{5}) the solution permutation 9237/46185 was
found. This is a solution, because the fraction it represents is equal to
goal-state's fraction.

\[\frac{9237}{46185} = \frac{3 \times 3079}{3 \times 5 \times 3079}
= \frac{1}{5}\].

This solution was found in just 6 steps listed here.

\begin{itemize}
  \item \textbf{root} state: 1234/56789 (0.02172956) 
  \item state: 9234/56781 (0.162624822)
  \item state: 9235/46781 (0.197409205)
  \item state: 9235/46187 (0.199948037)
  \item \textbf{goal} state: 9237/46185 (0.2)
\end{itemize}

\section{Checkers puzzle}

1. The initial state of the puzzle.
2. A general description of a goal state.
3. The method of assessing arc costs. It may be a xed value or a more complex procedure.
4. A clear, concise description (using mathematical expressions and text) of the heuristic function (h)
used to solve the puzzle.
5. A thorough description of the procedure used to generate successor states when expanding a node.
6. An overview description of a solution (i.e. path from start to goal) found by A*. The sequence of search
states from the start to the goal node must be presented along with a brief summary (in Norwegian or
English) of the main state-to-state transitions within sequence.


\end{document}
