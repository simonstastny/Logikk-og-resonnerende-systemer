\documentclass{article}

\usepackage[czech, english]{babel} \usepackage[T1]{fontenc} % pouzije EC fonty
\usepackage[utf8]{inputenc}
\usepackage{gensymb}
\usepackage{amssymb,amsmath}
\usepackage{graphicx}
\usepackage{xfrac}

\begin{document}

\providecommand{\abs}[1]{\lvert#1\rvert}

\title{Simulated Annealing - Report}
\author{Nicolas Boileau, Simon Stastny}

\maketitle

\section{Description of implementation}
Plain text\ldots

\section{Solutions found}

\subsection{Egg Carton puzzle K=2, M=5}

There can be 12 eggs placed in this variant of the puzzle.


X \_ \_ X \_ 

\_ X X \_ \_ 

X \_ \_ \_ X 

\_ \_ X \_ X 

\_ X \_ X \_ 

\subsection{Egg Carton puzzle K=2, M=6}

There can be 12 eggs placed in this variant of the puzzle.

\_ \_ \_ X X \_ 

X X \_ \_ \_ \_ 

\_ \_ \_ \_ X X 

X \_ X \_ \_ \_ 

\_ X X \_ \_ \_ 

\_ \_ \_ X \_ X 

\subsection{Egg Carton puzzle K=1, M=8}

There can be 8 eggs placed in this variant of the puzzle.

\_ \_ \_ \_ X \_ \_ \_ 

X \_ \_ \_ \_ \_ \_ \_ 

\_ \_ \_ \_ \_ \_ \_ X 

\_ \_ \_ \_ \_ X \_ \_ 

\_ \_ X \_ \_ \_ \_ \_ 

\_ \_ \_ \_ \_ \_ X \_ 

\_ X \_ \_ \_ \_ \_ \_ 

\_ \_ \_ X \_ \_ \_ \_ 

\subsection{Egg Carton puzzle K=3, M=10}

There can be 30 eggs placed in this variant of the puzzle. 

\_ X \_ \_ \_ \_ X \_ X \_ 

\_ \_ X \_ X X \_ \_ \_ \_ 

X \_ X \_ \_ \_ \_ X \_ \_ 

\_ X \_ \_ \_ \_ \_ X \_ X 

\_ \_ \_ X \_ \_ \_ \_ X X 

\_ \_ \_ X X \_ X \_ \_ \_ 

X X \_ \_ \_ \_ \_ \_ \_ X 

\_ \_ X \_ X \_ \_ X \_ \_ 

X \_ \_ X \_ X \_ \_ \_ \_ 

\_ \_ \_ \_ \_ X X \_ X \_ 

\section{Discussion}



\end{document}
