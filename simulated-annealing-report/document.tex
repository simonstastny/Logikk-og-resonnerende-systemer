\documentclass{article}

\usepackage[czech, english]{babel} \usepackage[T1]{fontenc} % pouzije EC fonty
\usepackage[utf8]{inputenc}
\usepackage{gensymb}
\usepackage{amssymb,amsmath}
\usepackage{graphicx}
\usepackage{xfrac}

\begin{document}

\providecommand{\abs}[1]{\lvert#1\rvert}

\title{Simulated Annealing - Report}
\author{Nicolas Boileau, Simon Stastny}

\maketitle

\section{Description of implementation}

We decided to implement just one of the puzzles: egg carton puzzle. This report
covers this only.

\subsection{Representation}

To represent our system, we use an array of booleans with 2
dimensions: boolean[x][y]. When there is an egg, the boolean is true. The
temperature is an integer that starts at 2000 and decrease slowly at a rate of
95\% per run until it reaches the minimal temperature of 1.

For each temperature, there are 20 iterations of finding solution and
accepting/rejecting it. This number is \texttt{ITERATION\_RUNS} parameter of
\texttt{SimulatedAnnealing} class and can be modified (however we found 20 to
be sufficient for this puzzle).

\subsection{Objective function}

To evaluate a solution, we first create a number that we consider as the
maximum number of eggs that we could possibly have in the box if we respect the
constraints. That doesn't mean that there is a solution in which we can put
that many eggs, but we are sure that we can't put more. This number is

\[
max = size\_of\_the\_box \times limit\_of\_eggs\_per\_line
\]

For example for a box with a size of 6*6 and a constraint of K=2, we consider
that we can never place more than 6*2 = 12 eggs.

Then, for a given solution, we count the \texttt{number of eggs in the box}, and
we retrieve the \texttt{number of violations}, which is the number of eggs that
are not respecting the contraints. For example if on a same line we have 4 eggs, with a
constraint K=2, we count 2 violations.

Finally, we retrieve this number to the first number we calculated. So for a
given solution X, we have a number calculated using the following formula:

\[
energy(X) = max - (eggCount() - violationCount());
\]

The smaller this number is, the closer we are from the optimal solution. If we
reach 0, then we are sure that we have an optimal solution. If there is at
least one violation, then the solution is not considered as valid. But we can
still adopt it as one of our node in our algorithm.


Now that we have our objective function, let be X our current best solution and
X' our new candidate. If X' is better (energy(X') < energy(X)), then we select
it as our new best solution. But even if it's not the best, we have a
probability to accept it. To decide whether we adopt a solution X' or not, we
generate the probability of accepting it at a temperature T using the following
formula:

\[
 P = \exp(\frac{1000 \times (energy(X) - energy(X'))}{T})
\]

Then we generate a random double between 0 and 1, if it's less than the
probability P we just calculated, then we accept the solution X' even though
it's not better than our current solution.

\subsection{Neighbour generation}

To generate a neighbour from a current state, we check whether the solution is
valid (no violation of the constraints) or not. If it is valid, we add one egg
randomly in a free space. If it is not valid, then we move one random egg to a
random free space. Thus, the difference between a state and one of its neighbour
will always be one egg added or moved.

\section{Solutions found}

\subsection{Egg Carton puzzle K=2, M=5}

There can be 12 eggs placed in this variant of the puzzle.

X \_ \_ X \_ 

\_ X X \_ \_ 

X \_ \_ \_ X 

\_ \_ X \_ X 

\_ X \_ X \_ 


\subsection{Egg Carton puzzle K=2, M=6}

There can be 12 eggs placed in this variant of the puzzle.

\_ \_ \_ X X \_ 

X X \_ \_ \_ \_ 

\_ \_ \_ \_ X X 

X \_ X \_ \_ \_ 

\_ X X \_ \_ \_ 

\_ \_ \_ X \_ X 


\subsection{Egg Carton puzzle K=1, M=8}

There can be 8 eggs placed in this variant of the puzzle.

\_ \_ \_ \_ X \_ \_ \_ 

X \_ \_ \_ \_ \_ \_ \_ 

\_ \_ \_ \_ \_ \_ \_ X 

\_ \_ \_ \_ \_ X \_ \_ 

\_ \_ X \_ \_ \_ \_ \_ 

\_ \_ \_ \_ \_ \_ X \_ 

\_ X \_ \_ \_ \_ \_ \_ 

\_ \_ \_ X \_ \_ \_ \_ 


\subsection{Egg Carton puzzle K=3, M=10}

There can be 30 eggs placed in this variant of the puzzle. 

\_ X \_ \_ \_ \_ X \_ X \_ 

\_ \_ X \_ X X \_ \_ \_ \_ 

X \_ X \_ \_ \_ \_ X \_ \_ 

\_ X \_ \_ \_ \_ \_ X \_ X 

\_ \_ \_ X \_ \_ \_ \_ X X 

\_ \_ \_ X X \_ X \_ \_ \_ 

X X \_ \_ \_ \_ \_ \_ \_ X 

\_ \_ X \_ X \_ \_ X \_ \_ 

X \_ \_ X \_ X \_ \_ \_ \_ 

\_ \_ \_ \_ \_ X X \_ X \_ 


\section{Discussion: heuristic and objective functions}

Both of these types of functions give a value to a solution to evaluate how good
it seems, but the \texttt{heuristic functions} require to know our goal in order
to evaluate how appropriate a candidate seems to be. We know how an optimal solution
looks like, so we can stop when we find it.

Whereas the \texttt{objective functions} evaluate a solution without knowing of
the optimal solution. They just give an estimation on how good it seems. It
allows us to compare two candidates in order to choose which one seems better
(in order to reach our goal), but we can't know if we have an optimal solution.

\end{document}
